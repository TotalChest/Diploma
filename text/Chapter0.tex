\section{Введение}
\label{sec:Chapter0} \index{Chapter0}

Операционная система Android является одной из самых востребованных на современном рынке мобильных устройств. Исследования показывают~\cite{clement2020number}, что на мобильном устройстве в среднем установлено от 60 до 90 приложений и пользователь тратит на приложения более двух часов в день. Следовательно, проверка надежности и корректности функционирования приложения -- важная задача. Как показывает статистика~\cite{packard2015failing}, большинство пользователей отказываются от использования приложения в случае его сбоя. Неправильное функционирование графического интерфейса не позволяет пользователям комфортно эксплуатировать весь потенциал приложений, который предоставляют разработчики. Ошибки приложения не только создают неудобства для пользователей, но и отрицательно сказываются на рейтинге продукта. Низкое качество Android-приложений можно объяснить недостаточным вниманием к тестированию в связи с быстрым развитием сферы~\cite{kochhar2015understanding}. Разработчики пренебрегают тестированием, поскольку этот процесс считается трудоемким, дорогостоящим и сопряженным с многократным повторением однотипных действий.

Избежать сбоев в работе мобильных приложений можно еще на этапе разработки, если основательно подходить к тестированию продукта. Тестировать приложения во время их разработки это один из ключевых элементов современного проектирования программного обеспечения. Причем необходимо делать это на всех этапах производства с помощью непрерывной интеграции и непрерывной доставки, иначе можно упустить мелкие ошибки и недочеты, которые в результате превратятся в негативный пользовательский опыт. Одним из основных видов проверки конечного продукта является тестирование приложения через взаимодействие с графическим интерфейсом. Этот вид тестирования проверяет отсутствие сбоев в работе приложения до того, как оно будет выпущено на рынок. Суть такого тестирования заключается в том, чтобы путем взаимодействия с разными интерактивными элементами посещать различные состояния приложения. Чем больше состояний будет покрыто во время тестирования, тем больше уверенность в том, что продукт исправен. Важно, что процесс тестирования происходит без доступа к исходному коду приложения.

Автоматическое тестирование приложений начинается с создания тестовых примеров, которые представляют из себя последовательности действий пользователя при взаимодействии с приложением. Разработка тестовых примеров обычно занимает много времени, так как производится вручную. Еще чаще тестирование проводится вовсе без написания тестовых примеров: ручное взаимодействие с устройством. Из-за разнообразия приложений на рынке и большого количества способов взаимодействия с ними такое тестирование становится дорогим и неэффективным. На практике можно столкнуться с тем, что тестирование занимает несколько часов, а небольшие изменения в приложении могут привести к новым ошибкам, что требует повторного тестирования~\cite{arnatovich2018systematic}. Поэтому компании часто выпускают бета-версии своих продуктов, чтобы тестированием занимались пользователи. 

За последние 10 лет было придумано и разработано много альтернативных инструментов, предназначенных для автоматизации данного тестирования~\cite{haoyin2017automatic, machiry2013dynodroid, studio2017ui, li2019deep, azim2013targeted, hao2014puma, mao2016sapienz}. Схема автоматического тестирования состоит из множества повторений одних и тех же действий: получение состояния устройства, генерация действия согласно алгоритму, выполнение действия на устройстве. Самым важным этапом является разработка алгоритма генерации нажатия, поскольку от него зависит общая эффективность процесса тестирования. Существующие подходы занимаются исследованием именно этого этапа. Однако оптимального инструмента тестирования все еще не существует~\cite{choudhary2015automated}. Причиной низкого покрытия тестирования является трудоемкость изучения сложных состояний приложения, достижимых только через конкретные последовательности действий. Исправить ситуацию можно с помощью интеллектуальных систем, основанных на алгоритмах машинного обучения. Одной из таких систем является недавно изобретенная связка тестирующей системы DroidBot~\cite{li2017droidbot} и нейросетевого алгоритма генерации действий Humanoid~\cite{li2019deep}.

Для измерения эффективности того или иного подхода используются различные метрики. Из самых важных показателей эффективности~\cite{memon2001coverage} можно выделить следующие:

\begin{itemize}

\item количество уникальных состояний, открытых во время тестирования

\item количество открытых Активностей приложения (соответствует принятому в англо-язычном сообществе термину Activity)

\item покрытие кода приложения

\item количество найденных сбоев и зависаний

\end{itemize}

Увеличение значений каждой из этих метрик способствует повышению эффективности тестирования приложений. Стоит упомянуть, что первые две метрики опираются на понятие состояния. Активность приложения -- некоторое обобщенное понятие состояния, введенное разработчиками Android. Важно понимать, что представление состояния влияет на эффективность тестирования. Используемое в этой работе представление будет описано в главе~\ref{sec:Chapter3}. 

В данной работе исследуются подходы, которые генерируют тестовые примеры на основе машинного обучения -- обучения с подкреплением~\cite{szepesvari2010algorithms}. Техника обучения с подкреплением использует метод проб и ошибок для выработки правильной последовательности действий, способной открывать новые состояния. Идея заключается в том, что каждое взаимодействие в каждом состоянии оценивается некоторой функцией награды. Функция награды формируется так, чтобы давать большее вознаграждение за полезные действия и штрафовать за бесполезные. В процессе исследования приложения алгоритм учится предсказывать какое действие приведет к большей суммарной награде и тем самым позволит исследовать как можно больше состояний.

Данная работа имеет следующую структуру. В главе~\ref{sec:Chapter1} дается формальная постановка задачи. В ней описываются цели работы и ограничения, которым должно удовлетворять решение. В главе~\ref{sec:Chapter2} делается обзор существующих решений для автоматического тестирования мобильных приложений. Затем рассматриваются работы, в которых применяются подходы обучения с подкреплением для решения похожих задач. В~\ref{sec:Chapter3}-ой главе проводится исследование поставленной задачи и построение алгоритмов ее решения, в том числе проверяются различные эвристики, потенциально способные повысить производительность тестирования. В этой же главе проводятся все необходимые эксперименты для определения лучшего из рассмотренных подходов. Затем лучший подход сравнивается с современным инструментом тестирования Humanoid~\cite{li2019deep}.