\section{Заключение}
\label{sec:Chapter5} \index{Chapter5}

В данной работе были исследованы современные методы динамического тестирования мобильных приложений через взаимодействие с графическим интерфейсом. Основное внимание было уделено изучению перспективных подходов, основанных на алгоритмах обучения с подкреплением.

В ходе работы были изучены основы функционирования алгоритмов обучения с подкреплением и популярный подход Q-learning. Были реализованы и протестированы существующие подходы для решения поставленной задачи. Также проводились эксперименты с изменением и улучшением существующих подходов, путем добавления некоторых эвристик, в том числе основанные на сверточных нейронных сетях. Были реализованы подходы способные тестировать целый набор приложений, используя одну модель обучения с подкреплением для всех приложений. Для повышения качества тестирования были реализованы алгоритмы, способные тестировать единственное приложение, под которое они обучаются. Также проводились эксперименты по внедрению идей из классического машинного обучения.

Результаты показывают, что подходы, предназначенные для тестирования одного приложения, превосходят обобщенные модели, способные тестировать набор приложений. Также добавление логичных эвристик даже в самые простые реализации способно повысить производительность тестирования. Можно сделать вывод, что не все современные подходы способны удовлетворять требованиям практического применения алгоритмов. В результате экспериментов удалось выделить алгоритм, основанный на предварительном обучении. Он значимо превосходит эвристические предположения.

В результате сравнения с современным подходом к тестированию Humanoid, лучший из методов обучения с подкреплением оказался существенно эффективней в рамках поставленной задачи.

Результаты этой работы будут использованы в промышленном продукте, разрабатываемом в Институте системного программирования им. В.П. Иванникова РАН. В перспективе можно рассмотреть влияние альтернативного представления состояний приложения. В том числе представление на основе графовых вложений.