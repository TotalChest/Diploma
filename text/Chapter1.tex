\section{Постановка задачи}
\label{sec:Chapter1} \index{Chapter1}

Основной целью данной работы является исследование методов обучения с подкреплением в задаче тестирования мобильных приложения на операционной системе Android через взаимодействие с графическим интерфейсом. Под тестированием в данной работе понимается проверка правильности функционирования готового приложения: отсутствие сбоев во время работы. Путем взаимодействий с графическим интерфейсом нужно обойти как можно больше состояний приложения и убедиться, что эти взаимодействия не вызывают сбоев.

Все рассмотренные подходы обучения с подкреплением должны быть внедрены в систему тестирования DroidBot~\cite{li2017droidbot}. Эксперименты, измеряющие эффективность тестирования должны быть выполнены многократно с последующим усреднением результатов. Выводы об эффективности того или иного подхода должны быть сделаны на основе следующих метрик качества:

\begin{itemize}

\item Количество уникальных состояний, посещенных во время тестирования;

\item Количество открытых Активностей приложения;

\end{itemize}

Все тесты должны проводиться на заранее выбранных приложениях. Приложения для тестирования должны быть различных категорий и разных степеней сложности.

Рассматриваемые подходы должны удовлетворять следующим требованиям:

\begin{itemize}

\item тестирование должно выполняться полностью автоматически, то есть без вмешательства пользователя;

\item тестирование не должно использовать исходный код приложения;

\item тестирование должно быть максимально приближено к человеческим взаимодействиям;

\item тестирование должно быть ограниченно по времени;

\end{itemize}

Третий пункт требований означает, что в работе не рассматриваются подходы, которые заведомо далеки от человеческих взаимодействий. Например, подходы с частыми случайными нажатиями. Последний пункт требований означает, что более эффективным подходом к тестированию в этой работе считается тот, который за фиксированное время показал наибольшее значение введенных ранее метрик.

Для чистоты эксперимента тестирование всех подходов должно проводиться на одной и той же ЭВМ.

Общий вывод об эффективности подходов обучения с подкреплением в задаче тестирования мобильных приложений должен быть сделан по результатам сравнения с тестирующей системой глубокого обучения Humanoid~\cite{li2019deep}.

