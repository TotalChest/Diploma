\section{Постановка задачи}
\label{sec:Chapter1} \index{Chapter1}

Основной целью данной работы является исследование методов обучения с подкреплением в задаче тестирования мобильных приложения на операционной системе Android. В работе рассматриваются Q-learning алгоритмы рабличной природы и с различными эвристиками. Все рассмотренные подходы должны быть внедрены в ситему тестирования DroidBot [...]. Эксперименты, замеряющие эффективность тестирования должны быть выполнены многократно с последующим усреднением результатов. Выводы об эффективности того или иного подхода должны быть сделаны на основе следующих метрик качества:

- Количество уникальных состояний, посещенных во время тестирования;

- Количество открытых Активностей приложения;

Все тесты должны проводиться на заранее выбранных приложениях. Приложения для тестирования должны быть различных категорий и разных степеней сложности.

Рассматриваемые подходы должны удовлетворять следующим требованиям:

- тестирование должно выполняться полностью автоматически, то есть без вмешательства пользователя;

- тестирование не должно использовать исходный код приложения;

- тестирование должно быть максимально приближено к человеческим взаимодействиям;

- тестирование должно быть ограниценно по времени;


Второй пункт требований означает, что в работе не рассматриваются подходы, которые заведомо далеки от человеческих взаимодействий. Например, подходы с частыми случайными нажаиями. Последний пункт требований означает, что более эффективным подходом к тестированию в этой работе считается тот, который за фиксированное время показал наибольшее тестовое покрытие.

Для чистоты эксперимента тестирование всех подходов должно проводиться на одной и той же ЭВМ.

Общий вывод об эффективности подходов обучения с подкреплением в задаче тестирования мобильных приложений должен быть сделан по результатам сравнения с тестирующей системой глубокого обучения Humanoid [...].

