\section{Описание практической части}
\label{sec:Chapter4} \index{Chapter4}
\todo[inline]{Если в рамках работы писался какой-то код, здесь должно быть его описание: выбранный язык и библиотеки и мотивы выбора, архитектура, схема функционирования, теоретическая сложность алгоритма, характеристики функционирования (скорость/память).}

Упомянуть:

- Язык Python (Совместимость с использованным инструментом DroidBot, скорость разработки, нужные библиотеки)

- библиотеки (в бильшинстве случаев хватало стандартной библиотеки, помимо этого numpy, pandas, matplotlib, zss, tensorflow/keras)

- UML диаграмма классов

- Последовательное описание алгоритма с выводом сложности временной и пространственной.

